
\newglossaryentry{ppu_g}{
    name={Picture Processing Unit},
    description={The part of the Gameboy that is responsible for rendering the game}
}
\newglossaryentry{ppu}{
    type=\acronymtype,
    name={PPU},
    description={Picture Processing Unit},
    first={Picture Processing Unit (PPU)\glsadd{ppu_g}},
}

\newglossaryentry{gb}{
    type=\acronymtype,
    name=GB,
    description={Gameboy},
    first={Gameboy (GB)}
}

\newglossaryentry{gbc}{
    type=\acronymtype,
    name=GBC,
    description={Gameboy Color},
    first={Gameboy Color (GBC)}
}

\newglossaryentry{os}{
	type=\acronymtype,
	name=OS,
	description={Operating System},
	first={Operating System (OS)}
}

\newglossaryentry{cpu}{
    type=\acronymtype,
    name=CPU,
    description={Central Processing Unit},
    first={Central Processing Unit (CPU)}
}

\newglossaryentry{apu_g}{
    name={Audio Processing Unit},
    description={The part of the Gameboy that is responsible with processing the game's audio}
}

\newglossaryentry{apu}{
    type=\acronymtype,
    name=APU,
    description={Audio Processing Unit},
    first={Audio Processing Unit (APU)}
}

\newglossaryentry{rom}{
    type=\acronymtype,
    name={ROM},
    plural={ROMs},
    description={Read-Only Memory},
    first={Read-Only Memory (ROM)},
    firstplural={Read-Only Memory (ROM)}
}

\newglossaryentry{mbc}{
    type=\acronymtype,
    name={MBC},
    plural={MBCs},
    description={Memory Bank Controller},
    first={Memory Bank Controller (MBC)},
    firstplural={Memory Bank Controllers (MBCs)}
}

\newglossaryentry{oam}{
    type=\acronymtype,
    name={OAM},
    description={Object Attribute Memory},
    first={Object Attribute Memory (OAM)},
}

\newglossaryentry{vram}{
	type=\acronymtype,
	name={VRAM},
	description={Video RAM},
	first={Video RAM (VRAM)}
}

\newglossaryentry{wram}{
	type=\acronymtype,
	name={WRAM},
	description={Work RAM},
	first={Work RAM (WRAM)}
}

\newglossaryentry{dma}{
    type=\acronymtype,
    name={DMA},
    description={Direct Memory Access},
    first={Direct Memory Access (DMA)},
}

\newglossaryentry{dmg_g}{
    name={Dot Matrix Game},
    description={The model name of the Gameboy. It is often used to refer to the ``base'' Gameboy (in contrast with GBC for the Gameboy Color)}
}
\newglossaryentry{dmg}{
    type=\acronymtype,
    name={DMG},
    description={Dot Matrix Game},
    first={Dot Matrix Game (DMG)\glsadd{dmg_g}},
}

\newglossaryentry{tcyc}{
    name={T-Cycle},
    description={A T-cycle is the smallest step the internal clock of the Gameboy can do. This means the rate of T-cycle is that of the CPU, ie. 4.19Mhz}
}

\newglossaryentry{mcyc}{
    name={M-Cycle},
    description={An M-Cycle (or machine cycle) is the smallest step the CPU of the Gameboy can do. Because all instructions of the CPU are multiples of 4, instruction lengths and timings are usually referred to in M-cycles (e.g. \texttt{LD A, B} takes 4 T-cycles, thus 1 M-cycle)}
}

\newglossaryentry{mmap}{
    name={Memory Map},
    description={The memory map is what determines where each address leads to - it can be seen as a list of non-overlapping ranges}
}

\newglossaryentry{msb}{
    type=\acronymtype,
    name={MSB},
    description={Most Significant Bits},
    first={Most Significant Bits (MSB)},
}
